\documentclass[twocolumn]{article}
	\usepackage[
	top    = 3 cm,
	bottom =  3 cm]{geometry}

	\usepackage{hyperref}

	\usepackage{graphicx}  % needed for figures
	\usepackage{dcolumn}   % needed for some tables
	\usepackage{bm}        % for math
	\usepackage{amssymb}   % for math
	\usepackage{amsmath}
	\usepackage{nameref}
	\newcommand{\degree}{\ensuremath{^\circ}}
	\def\mean#1{\left< #1 \right>}
	%%% make INLINE SUBSECTIONS 
		
		% 1em is in the 'sep' slot, which controls how much space is between the numbering and the section heading text. 1em is equal to the width of 1 times the width of a capital 'M' in font you're using. 
	\usepackage{titlesec}
	\titleformat{\subsection}[runin]
	{\normalfont\large\bfseries}{\thesubsection}{1em}{}
		% The formatting settings above other than the 'runin' option are the default settings for subsections. 


	\begin{document}
	\title{Problem set }
	\author{Giulio Ungaretti}
	\date{\today}
	\twocolumn[
	   \begin{@twocolumnfalse}
	     \maketitle
	     \begin{abstract}
	       Problem set solutions.
	     \end{abstract}
	     \tableofcontents
	   \end{@twocolumnfalse}
	  ]

\clearpage
\section{Distributions \& \\ probabilities} % (fold)
	\label{sec:Distributions and probabilities}
	\subsection{} % (fold)
	\label{sub:}
	% subsection  (end)
	Given the PDF:
	\begin{equation}
		\operatorname{f}{\left (t \right )} = C e^{- \frac{t}{\tau}}
	\end{equation}
	in the range $t$ $\in~$[$ t_0, \infty $], the value of C for which the PDF is normalized is:

	\begin{equation}
		\int_{t_{{0}}}^{\infty} C e^{- \frac{t}{\tau}}\, dt = 1 \longrightarrow C =  \frac{
		e^{
		\frac{t_0}{\tau}
		}}
		{\tau}
	\end{equation}
	I note a close resemblance of our PDF to the exponential distribution, and they coincide if $t_0 = 0$. 
	The mean, or better, the expectation  value is then:
	\begin{equation}
	\mathbb{E} [t] = \int_{t_0}^{\infty} t \operatorname{f}{(t) d t } \longrightarrow  t_0 + \tau
	\end{equation}
	It's nice to note that for $t_0 = 0$ it matches the well known result for the exponential distribution.
	The width could be quantified with the variance as:
	\begin{multline}
		Var(t) = \mathbb{E}[x^2] - (\mathbb{E}[x])^2 = \\
		= 
		\int_{t_0}^{\infty} t^2 \operatorname{f}{(t) d t } 
		-(\int_{t_0}^{\infty} t \operatorname{f}{(t) d t })^2 = \\
		= 2 \tau ^ 2 + 2 t_o \tau +t_0 ^2  - ( t_0 + \tau ) ^ 2 = \\
		= \tau ^ 2
	\end{multline}

	\begin{figure}[htb]
		\begin{center}
			\includegraphics[width= 0.45 \textwidth]{fig/graph.pdf}
		\end{center}
		\caption{Plot of the given PDF, for different values of $ \tau $ and $t_0 $. Dashed lines are the calculated expectation values, and the shaded area represent plus and minus one square root of the variance around the mean.}
		\label{fig:pdf}
	\end{figure}
	A graphical representation of all said above is found in Figure~\ref{fig:pdf}.
	\subsection{} % (fold)
	\label{sub:}
	Assuming that little Peter is a fair pal, and is using fair coins then the probability of success (heads) follows the binomial distribution.
	If we call $n$ the number of tries, $p= \frac{1}{2}$ the probability for each event to success, and $r$ the number of success out of $n$ tries, the probability is the following: 
	\begin{multline}
	\label{eq:bin}
		P (n,r) = \frac{n!}{r!(n-r)!} p ^ r (1-p)^{n-r} \\ 
		 = \tbinom n r p ^ r (1-p)^{n-r}
	\end{multline}
	The \emph{chance} of getting 14 or more heads with 20 coin flips is \emph{low}, as it is possible to see in~Figure~\ref{fig:lil}. The probability of getting more than 14 heads is calculated summing equation~\ref{eq:bin} for r going from 14 to 20.
	The probability is  $\approx 5 \% $.

	\begin{figure}[h]
		\begin{center}
			\includegraphics[width=.45 \textwidth]{fig/lil.pdf}
		\end{center}
		\caption{Probability of coin flips. The parameters are reported in the legend; the shadowed area quantifies the chance of getting 14 or more heads.}
		\label{fig:lil}
	\end{figure}


	To find the chance that little Peter gets at least 18 coin at once when flipping 20 coins 100 times, first the probability $P_{18}$ is calculated summing equation \ref{eq:bin} using $p= \frac{1}{2}$, $n=20$, and $r=[18,20]$.
	It is easy to see that the distribution of the chances of getting at least 18 successes is still binomial but with p = $P_{18}$, $n=100$, and $r=1$.
	The probability of getting one time a success is: $P(1) \approx 0.0002$. The chance of getting more than one is quantified as said before and it is  $\approx 2 \% $.
% subsection  (end)
% section Distributions and probabilities (end)
\clearpage
\section{Error propagation} % (fold)
\label{sec:error_propagation}
\subsection{} % sub:gravity (fold)
	Three different ways are employed to calculate the average of the given measurements, namely the arithmetic mean, a error weighted mean and finally a fit with a zeroth order polynomial.
	The last two options require to accept and believe in the reported experimental errors.
	The results are reported in Figure~\ref{fig:g}.

	\begin{figure}[h!]
		\begin{center}
			\includegraphics[width=.4\textwidth]{fig/g.pdf}
		\end{center}
		\caption{Graphical presentation of the given measurements and three differently calculated averages. The shadowed areas around each average value are plus and minus $\sigma_{\mu}$, the associated error.}
		\label{fig:g}
	\end{figure}

	The $\chi ^2 $ probability density function is reported in Figure~\ref{fig:xpdf} along with the probability of getting the same $\chi ^2 $ found in the aforementioned zeroth order polynomial fit.

	\begin{figure}[h!]
		\begin{center}
			\includegraphics[width=.4 \textwidth]{fig/xpdf.pdf}
		\end{center}
		\caption{$\chi^2$ distribution for five degrees of freedom. Shaded area represent the probability of obtaining a $\chi^2$ greater the one obtained for the fit reported in Figure~\ref{fig:g}.}
		\label{fig:xpdf}
	\end{figure}
	The result obtained are summarized in Table~\ref{tab:res}.

	\begin{table}[htpb]
		\caption{Results. The averages and their errors are reported in units of: $10^{-11} m^3 kg^{-1} s^{-2}$.}
		\label{tab:res}
		\begin{center}
			\begin{tabular}{l|ccc}
			\hline

			\hline
			\textbf{method} & \textbf{average} & \textbf{uncertainty} & \textbf{$\chi ^2 $(prob)} \\
			\hline
			fit & 6.5 & .2 & 12.261 (0.031) \\
			$<G> $ & 6.9 & .3 & - \\
			$<G>_W$ &  7.1 & .2 & - \\
			\hline
			\hline
			\end{tabular}
		\end{center}
	\end{table}

	% section error_propagation (end)
	As it possible to see the fit is the method that is better representing the data with an average value.
	Moreover it is also compatible with the \emph{true} value of G ($ G = 6.6738 \pm .0008  \times 10^{-11} \ \mbox{m}^3 \ \mbox{kg}^{-1} \ \mbox{s}^{-2} $).
	The probability of the fit is not particularity high but is expected due to the low number of measurements.
	Lastly it is possible to note that the probability to get $\chi ^2 \gtrsim 12.261 $ is $ \approx 3 \% $ according the the $\chi ^2$ distribution (Figure~\ref{fig:xpdf}, shaded area).
\subsection{} % (fold)
\label{sub:gridiron}
	The uncertainty on a time measurement with a gridiron pendulum is:
	\begin{equation}
		\sigma_t  = 2 \pi \sqrt{ (\frac{1}{2 \sqrt{g L }} ) ^{2} \sigma_L ^2 +
		\sigma_g ^2 (\frac{\sqrt{L}}{2} g^{-\frac{3}{2}})^2
		}
	\end{equation}
	where L is the length of the pendulum, g is the gravitational constant with an error $\sigma_g$.
	If one considers that the error on the measurement on the length is \textbf{only} related to the thermal expansion of the materials making the pendulum then $\sigma_L$ can be calculated as:
	\begin{equation}
		\sigma_L = \sum_i \alpha_i  \Delta T  L_i
	\end{equation}
	where $\alpha_i$ is the linear coefficient of expansion for the \emph{ith} material, $L_i$ its length and $\Delta T $ the temperature fluctuation of the system.

	If the pendulum is made of a single component:
	\begin{equation}
	\sigma_t  = \pi \sqrt{L \left(\frac{\Delta T^{2} \left(a_{1} \right)^{2}}{g} + \frac{\sigma_{g}^{2}}{g^{3}}\right)}
	\end{equation}
	whereas if the pendulum is made of two components with a length ratio of $\lambda$:
	\begin{equation}
	\sigma_t  = \pi \sqrt{L \left(\frac{ \Delta  T^{2} \left(a_{1} - a_{2} \lambda \right)^{2}}{g} + \frac{\sigma_{g}^{2}}{g^{3}}\right)}
	\end{equation}

	It's easy to see that the error coming from the thermal expansion can be killed by choosing the correct material combination such that $ \left(a_{1} - a_{2} \lambda\right) \approx 0 $. Strangely enough that's the case for the iron/brass combo.
	% subsection  (end)
\subsection{} % (fold)
\label{sub:snell}
	To calculate the index of refraction (IOR) of a solution ($n_{sol}$) with respect to air one has to rearrange Snell's law and to calculate the error ($\sigma_{n_{sol}}$) one has to consider the IOR of air a value ($n_{air} =1 $) without error.
	\begin{equation}
	n_{sol} = \frac{ sin(\theta_{air})}{sin(\theta_{sol})} 
	\end{equation}

	\begin{multline}
	\sigma_{n_{sol}} = (\left[\frac{cos(\theta_{air})}{sin(\theta_{sol})}  \sigma_{air} \right]^2 + \\
	+ \left[ \frac{2 cos(\theta_{sol})}{1+cos(\theta_{sol})-1}  sin(\theta_{air}) \sigma_{sol}  \right] ^2 )^{\frac{1}{2}}
	\end{multline} 
	To determine the percentage and its error of sugar in the solution, using two known and error-less standard $n_1 = 1.3330$, and $n_2 = 1.4774$ for a  0\% and a 75\% solution respectively one has to use the following equations:
	\begin{equation}
	\% sugar = ( n_{sol} - 1.3330) \frac{75}{ (1.4774- 1.3330)}
	\end{equation}
	\begin{equation}
		\sigma_{\%} = \frac{75}{ (1.4774- 1.3330)} \sigma_{n_{sol}}
	\end{equation}
	The results are reported in Table~\ref{tab:snell}.
	\begin{table}[h!]
		\caption{Snell's law results}
		\label{tab:snell}
		\begin{center}
			\begin{tabular}{l|cc}
			\hline

			\hline
			\textbf{} & \textbf {$n_{sol}$ } & \textbf{ \% sugar} \\
			\hline
				value & 1.385  & 27 \\
				error & 0.008 & 4 \\
			\hline

			\hline
			\end{tabular}
		\end{center}
	\end{table}
	% subsection  (end)

\section{Monte Carlo} % (fold)
\label{sec:monte_carlo}
\subsection{}
 Having a PDF:
 \begin{equation}
 \label{eq:mcpdf}
 	\operatorname{f}{\left (x \right )} = \frac{C}{x^{3}}
 \end{equation}
 normalized for the value of $C=2$.
 We want to generate numbers that follow this PDF. 
 The transformation method would  work as the integral of the PDF:
 \begin{multline}
 	\int_{1}^{x(r)} \frac{2}{x^{3}}\, dx = 1 - \frac{1}{x(r)^{2}} = r
 \end{multline}
	is  invertible, given that $r \in [0,1[$. 
	A thousand number according to the transformation method are calculated and they are reported in 

	\begin{figure}[h!]
		\begin{center}
			\includegraphics[width = .4 \textwidth]{fig/monteCPDF}
		\end{center}
		\caption{One thousand number following the PDF reported in equation \ref{eq:mcpdf},generated via a Montecarlo Method.}
		\label{fig:figure1}
	\end{figure}


\section{Statistical tests} % (fold)
\label{sec:statistical_tests}

\subsection{}
	The distribution of HT signals $N_{HT}$ for electrons follows a binomial distribution where the number of tries is the number of straws and the probability of each try (straw) is 22.6 \%.
	\begin{figure}[h]
		\begin{center}
			\includegraphics[width= .4\textwidth]{fig/trt.pdf}
		\end{center}
		\caption{Bar Plot depicting the distribution of the HT signals for electron. The distribution is a binomial distribution with the parameters reported in the legend.}
		\label{fig:figure1}
	\end{figure}
	
	A test requiring $N_{HT} \ge 6$ has an electron detection efficiency of 83.6 \%, and a pion efficiency of 0.3 \%.
% section monte_carlo (end)



% section statistical_tests (end)


\section{Fitting data} % (fold)
\label{sec:fitting_data}

\section{notes}
A git-hub repo with all the source code is available at:
\url{http://giulioungaretti.github.io/stats2013}
% section fitting_data (end)
\end{document}